% ┌───────────────────────────────────────────────────────────────┐
% │ Contents of resume.tex                                        │
% ├───────────────────────────────────────────────────────────────┘
% │
% ├── PREAMBLE
% ├──┐BODY
% │  ├── HEADER
% │  └──┐CONTENT
% │     ├── EDUCATION
% │     ├── EXPERIENCE
% │     └── ADDITIONAL INFORMATION
% │
% └───────────────────────────────────────────────────────────────

% ################################################################ PREAMBLE

\DocumentMetadata{
 testphase=phase-II,
 pdfversion=2.0,
 pdfstandard=A-4
}

% variables
\author{Erica Ruggiero}
\title{Erica Ruggiero Résumé}

\documentclass[a4paper,11pt]{business}

\setmainfont{Minion 3}

% ################################################################ BODY

\begin{document}

% ################################ HEADER

\avatar{photo}
\name{Erica Ruggiero}
\city{Cambridge, MA 01242}
\mobile{(413) 464-4238}
\mail{Erica.Ruggiero@sloan.mit.edu}
\website{https://linkedin.com/in/erikaruggiero}

\thispagestyle{text}
%\thispagestyle{image}

% ################################ CONTENT

% ################ EDUCATION

\section{Education}

\entry
 {MIT Sloan School of Management}
 {Candidate for MBA, June 2017}
 {2015}
 {Present}
 {Cambridge, MA}{
 \begin{itemize}
  \item Enterprise Management Track: Consulting for national Internet service provider on Smart City technology strategy
  \item Volunteer Lead and Organizing Committee Member for Breaking the Mold Conference, Sloan Women in Management
  \item Active member of the Management Consulting and Technology Clubs
 \end{itemize}
}

\entry
 {Cornell University}
 {BS in Electrical and Computer Engineering, magna cum laude}
 {2008}
 {2012}
 {Ithaca, NY}{
 \begin{itemize}
  \item GPA: 3.92/4.3, Dean's List seven semesters
  \item Chair of the Society of Women Engineers Outreach Committee
 \end{itemize}
}

% ################ EXPERIENCE

\section{Experience}

\entry
 {Intel Corporation}
 {Digital Design Engineer (Promoted from Grade 3 to Grade 5 in 18 months, 6 months early)}
 {2011}
 {2015}
 {Hudson, MA}{
 \begin{itemize}
  \item Designed and led development for a test chip to be used in corporation-wide process technology research; design was the most complex to be implemented on a test chip in team’s history
  \item Influenced management to supply a multi-discipline team of three engineers to assist in test chip project execution in order to meet tight 6-month deadline and ensure highest quality standards
  \item Built consensus between team and internal stakeholders who commissioned the test chip through weekly discussions about feasibility of chip features; led a rigorous final design review with customers and senior technical leadership
  %\item Managed coordination of key design tasks and milestones for the test chip, delegated work to team members and initiated bi-weekly team syncs so that the team was able to deliver the chip to our partners ahead of manufacturing deadline
  \item Persuaded multiple managers to expand personal responsibilities to a dual logical and physical design role; reduced time to converge design by 30\%, increased cross-functional knowledge sharing, and piloted dual role for future projects
  %\item Created a design-automation environment from scratch that enabled test chip engineers to iterate on designs in hours instead of days, saving a cumulative 3 weeks of engineering time across the project lifetime
  \item Streamlined code-generation process that resulted in 50\% reduction in engineering hours per week; maintained the code database which was used by over 60 engineers across three Intel projects
  %\item Extracted meaningful data and trends from thousands of lines of design data per day using existing software and personal scripts, and analyzed data to inform and prioritize design decisions
  \item  Converged functional unit blocks to design milestones across four Intel Xeon microprocessors and Omni-Path products; ramped up quickly on each new project to meet tight turnaround time of one to two months between milestones
  \item Co-led the Women at Intel Network Hudson chapter: organized volunteer opportunities, professional development workshops and networking events, negotiated with corporate and site groups for over \$6K in funding
  %\item Organized annual on-site Women at Intel Network conference attended by over 100 employees, 20\% of the Hudson site
  %\item  Formalized organization’s intern program: set expectations for all summer interns, including team-building activities, review procedures, cross-team training and end-of-term presentations
  %\item Mentored two interns, actively involved in university recruiting and interviewing
 \end{itemize}
}

\entry
 {Intel Corporation}
 {Physical Design Intern}
 {2010}
 {2011}
 {Hudson, MA}{
 \begin{itemize}
  \item Modeled timing and power consumption for chip critical paths and proposed optimal routing schemes for those paths
  \item Developed chip power plan after pooling input from cross-site team members; plan implemented across all chip SKUs
 \end{itemize}
} 

% ################ ADDITIONAL INFORMATION

\section{Additional Information}

\begin{itemize}
 \item Patent holder, US Patent 9,138,339: a tool gripper device that adapts items like eating utensils to a larger handle circumference, allowing people with handicaps to easily maneuver those items; developed tool gripper in High School
 \item Budget Buddies Coach: taught financial literacy to low-income women from the community (2013-2015)
 \item Interests: Amateur painter and hiker, seasoned traveler (Asia, Europe and Americas regions)
\end{itemize}


\end{document}
